%!TEX program = xelatex
%使用xelatex,使用unicode,对中文支持的最好.记得文档另存为uft-8格式.

\documentclass[10pt,a4paper,titlepage]{article} %默认字体大小,页面大小(a4),标题单独一页
\usepackage{fontspec}   %引入设置字体的包
\usepackage{xeCJK}  %使用中文环境的包,还有一个是CJK,未知

\setmainfont{Times New Roman}   %英文字体
\setCJKmainfont[BoldFont=MicrosoftYaHei]{MicrosoftYaHei}      %中文字体`'


\title{资源管理器概要设计}
\author{作者:闫思维}
\date{2018.8.30}

\begin{document}
	\maketitle
	\newpage

	\begin{flushleft}
	\section{概述(Introduction)}
		\subsection{目的(Purpose)}
		本文主要是TVOS中间间的子模块资源管理器模块的概要设计,重点描述了如何
		通过资源管理器使用,管理独占的硬件资源,以满足PIP等多个播放器同时播放
		的需求场景。
		\subsection{范围(Scope)}
		本文主要包括RM的接口设计及使用场景设计,由于模块本身较小,实现较简
		单,不再做内部分解设计。
		\subsection{略略语(Acronyms \& Definaitions)}
		\begin{tabular}{|c|c|c|}
		\hline
		缩略语&英文&中文解释\\
		\hline
		TVOS&TVOS&TCL智能电视中间件系统\\
		\hline
		RM&Resource Manager&资源管理器,TVOS中间件的子模块。\\
		\hline
		\end{tabular}

		\subsection{参考(References)}
		SITA中间件概要设计DTVM中间件规范
		\subsection{发布范围}
		\begin{tabular}{|c|c|p{150pt}|c|}
		\hline
		序号&持有人角色&持有人姓名&发布日期\\
		\hline
		1&作者&李辉&2016.7.6\\
		\hline
		2&初评&樊二锋、赵德民、罗阳志、路惠明、付勇、林舜大、黄高波等&2016.7.8\\
		\hline
		\end{tabular}
		\newpage
	\end{flushleft}

\end{document}
